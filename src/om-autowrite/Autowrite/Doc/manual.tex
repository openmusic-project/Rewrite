\documentclass[11pt]{llncs}
\usepackage{a4wide}
\usepackage{amsmath}
\usepackage{amsfonts}
\usepackage{amssymb}
\usepackage{psfig, latexsym}
\usepackage{epsfig}
\usepackage{moreverb}
\newcommand{\A}{\mathcal{A}}
\newcommand{\BB}{\mathcal{B}}
\newcommand{\CC}{\mathcal{C}}
\newcommand{\DD}{\mathcal{D}}
\newcommand{\RR}{\mathcal{R}}
\newcommand{\LL}{\mathcal{L}}
\newcommand{\RRa}{{\RR_\alpha}}
\newcommand{\Sc}{\mathcal{S}}
\newcommand{\FF}{\mathcal{F}}
\newcommand{\GG}{\mathcal{G}}
\newcommand{\TT}{\mathcal{T}}
\newcommand{\NF}{\text{NF}}
\newcommand{\ENF}{\text{ENF}}
\newcommand{\CBN}{\text{CBN}}
\newcommand{\CBNs}{\text{CBNs}}
\newcommand{\WN}{\text{WN}}
\def\autocom#1{\textsl{#1 }}
\def\autocombf#1{{\bf \textsl{#1}}}
\begin{document}
\bibliographystyle{plain}
\title{Autowrite: 
%A tool for checking properties of term rewrite systems\\
User's Guide\\
\small{(Still being written)}}
%\author{Ir\`ene Durand\inst{1}}
\author{Ir\`ene Durand}
\institute{
Universit\'e de Bordeaux I \\
33405 Talence, France \\
\email{idurand@labri.fr}
}
\date{\today}
\maketitle 
\section{Introduction}
Autowrite is a tool for handling term rewrite
systems and tree automata. 
It was originally designed to check membership to Call-by-need 
($\CBN$) classes.
For this purpose, it implements the tree automata constructions used in
\cite{J96,DM97,DM98,NT02} and many useful
operations on terms, rewrite systems and tree automata.
Now all these automata constructions are accessible from the graphical
interface which makes Autowrite also a tool for handling tree automata.

The graphical interface (still under construction)
is written using FreeCLIM, the free implementation
of the CLIM specification. 
New functionnalities could be easily integrated upon demand.

The Autowrite tool was used to check membership to $\CBN_\alpha$ for most 
of the examples presented in \cite{DM01} and \cite{id-am:ic-2005}.
The latest version was presented in \cite{D04}.

\section{Installation}
The system has been compiled to run on a PC x86 under Linux.
The graphical interface is an X11 client so requires an X11 server.

The user should get the file Autowrite.tgz and install the system
with the following commands.
\begin{verbatim}
tar xzvf Autowrite.tgz
cd Autowrite
./INSTALL
\end{verbatim}
The system is started by typing
\begin{verbatim}
./autowrite &
\end{verbatim}
Use the {\tt Quit} button to exit Autowrite.

The directory {\tt Autowrite} contains a {\tt Data} directory were
the specifications are stored. 

\section{Specification Files}
The user may specify a set of Autowrite objects (Trss, automata, termsets)
related to a common signature and set of variable, in a specification file whose name 
should have the extension \texttt{.txt}.

An Autowrite specification file starts with the definition
of a signature, eventually followed by the definition of a set of variables.
Next we may have in any order definitions of TRSs, automata, termsets,
each one of them associated with a distinct name.

\pagebreak
\noindent
Example: \texttt{WRS.txt}
\begin{verbatim}
Ops 0:0  s:1 +:2 *:2
Vars x y
TRS R
; addition
+(0,x) -> x
+(s(x),y) -> s(+(x,y))
; product
*(0,x) -> 0
*(s(x),y) -> +(*(x,y),y)

Automaton EVEN
States odd even
Final States even
Transitions
0 -> even
s(even) -> odd
s(odd) -> even

Termset RS 0 s(x)
Termset "T(F)" x

Term *(*(0,s(0)),+(0,s(0)))
Term *(o,+(0,s(0)))
Term *(*(0,s(0)),o)
Term s(s(s(0)))
\end{verbatim}

%\verbatimtabinput{WRS.txt}

\subsection{Comments}
Any part of a line located after the character \texttt{;} 
is considered a comment and will be ignored.

\subsection{Names}
A name used for a symbol, a variable, a state, a trss or an automaton
should not contain the following characters \texttt{;, :, (, ), ", -}
unless the name is surrounded by \texttt{" "}.
The names \texttt{o} and \texttt{@} are reserved (\texttt{o} represents
the $\bullet$ symbol and \texttt{@} is the extra symbol used to extend
a signature.

\subsection{Symbols and signature}
A \emph{symbol} has a \emph{name} and an \emph{arity}
 separated by the character \texttt{:}.\\
A \emph{signature} is introduced by the \texttt{Ops} keyword followed by
a list of symbols.\\

\subsection{Variables}
The \emph{variables} are introduced by the \texttt{Vars} keyword followed by
a list of variables names which should be disjoint from the
symbol names.

\subsection{Terms}
A \emph{term} is introduced by the \texttt{Term} keyword followed by the term.
Terms are represented in the usual infix notation and
must be built upon the signature and the set of variables.

\subsection{Trss}
A \emph{trs} is introduced by the \texttt{TRS} keyword followed
by the name of the TRS and by the list of rules.

Each \emph{rule} has a left-handside and a right-handside separated
by \verb+->+. Both the left-handside and the right-handside
are terms.

\subsection{Automata}
An \emph{automaton} is introduced by the \texttt{Automaton} keyword followed
by the name of the automaton. Follow in that order states,
final states and transitions.\\
%
The \emph{states} are introduced by the \texttt{States} keyword followed
by a list of the states names.\\
%
The \emph{final states} are introduced by the \texttt{Final States} keyword 
followed by a list of the final states names.\\
%
The \emph{transitions} are introduced by the \texttt{Transitions} keyword 
followed
by a list of transitions.
Each transition has a left-handside (a flat term with arguments
being states) and a right-handside (a state) separated by \verb+->+.

\subsection{Termsets}
A \emph{termset} is introduced by the \texttt{Termset} keyword followed
a list on terms (not necessarily ground).

\section{Using Autowrite}

\subsection{Generalities}
The top pane of the Autowrite window is an interactor pane from which the user interacts with
the system. It prompts the user for commands and arguments.

At any time the user can get \textbf{help} about the current possibilities
by clicking on the right-button.

Commands are accessible either thru the command line using \textbf{completion} 
(achieved with the \verb+<TAB>+ key
or from the menus items. Some commands are also available from buttons.

All commands accessible are accessible by typing
the command name (written on the item) to the \autocom{Command} prompt.
The latter is recommended as thru completion it will save the user a lot
of typing. Completion works for commands, data filenames, automaton names,
trs names, termset names.

The menu are there to remind the user of every possible command or for people
who don't like to type. In any case, at the moment, arguments of commands
need to be typed in the interactor pane.

The \textbf{Read} commands are used to read Autowrite objects into the current specificationc directly
from the command line. 
The \textbf{Load} commands are used to load Autowrite objects into the current specificationfrom a file.
The \textbf{Save} commands are used to save Autowrite objects into a file.
The \textbf{Retrieve} commands are used to set the current Autowrite object
with an already computed object.

\subsection{File Menu}
From this menu, one can load a specification from a file or save the current
specification to a file.

\subsection{Trs Menu}
This menu gathers all commands related to the current TRS or to the left-handsides
of the current TRS.
\subsubsection{Left-linearity}
One can check left-linearity for the current TRS $\RR$ 
by using the \autocom{Left-linearity} command.
\subsubsection{Overlappingness}
One can check overlappingness for the current TRS $\RR$
by using the \autocom{Overlapping} command.
\subsubsection{Orthogonality}
One can check orthogonality for the current TRS $\RR$ 
by using the \autocom{Orthogonality} command.
\subsubsection{Normal and external normal forms}
One can check whether the set of normal forms $\NF_\RR$ or the set of
external normal forms $\ENF_\RR$ are empty using the \autocom{NF empty?}
and \autocom{ENF empty?} commands.
When the set is not empty, a term witnessing nonemptyness is shown.
\subsubsection{Forward-branching}
One can check whether a system is forward-branching~\cite{D94}
by using either the cubic algorithm derived from the characterization of
forward-branching system or the quadratic construction of a
forward-branching index-tree.
One can transform a forward-branching TRS to a constructor system in
$\CBNs$ using the two algorithms presented in \cite{bs-ris:jsc-simulation}.

\subsection{Approx Trs Menu}
This menu gathers all commands related to the current approximation
of the current TRS $\Sc = \RR_\alpha$. 

\subsubsection{Changing the current approximation}
With the \autocom{Approximation Strong} 
\autocom{Approximation Nv} 
\autocom{Approximation Linear Growing} 
\autocom{Approximation Growing} 
commands,
one can select the current approximation $\alpha$ for the current TRS $\RR$.
One may also use the \autocom{Approximation} item of the
\autocom{Approx Trs} menu and select in the pop-up-menu the desired
approximation.

\subsubsection{Membership to $\CBN$}
One can check membership to $\CBN_\alpha$ for the current TRS $\RR$ 
and the current approximation $\alpha$ by using the \autocom{Call-by-need} 
command.

The answer is either ``the TRS is in $\CBN_\alpha$ or
``the TRS is not in $\CBN_\alpha$; in the latter case an 
``$\alpha$-free-term''
\emph{i.e.} a term with no $\RR_\alpha$-needed redex is shown.

The \autocom{Call-by-need\@} command checks membership to $\CBN_\alpha$ 
but with the signature extended with a fresh constant symbol \autocom{\@}.
Indeed a TRS may be in $\CBN_\alpha$ with the signature consiting of 
all the symbols appearing in its rules but {\bf not} in $\CBN_\alpha$ with 
the same signature extended by just one fresh symbol. The preservation of 
membership thru ``signature extension''
has been investigated in \cite{DM01}.

Note that the membership to $\CBN_\alpha$ problem 
having a double exponential time complexity,
may run for a very long time for big TRS.\\
Use the \autocom{Stop} button to stop the computation.

\subsubsection{Preservation of the set of normalizable terms}
As shown in \cite{DM01}, the sets of terms of $\TT(\FF)$ that are normalizable
may be augmented if the signature $\FF$ is extended with just one fresh constant
symbol (for instance \@).
Using the command \autocom{WN(S,G,F) = WN(S,F)} one can check whether signature
extension preserves the set of normalizable terms. If this is not the case,
then a term witnessing that $\WN(\RR_\alpha,\GG,\FF) \not\subseteq \WN(\RR_\alpha,\FF)$ is shown.

The same can be done with $\RR \cup \{ \bullet \to \bullet \}$ as for the growing
approximation we may have 
$\WN(\RR_\alpha,\GG,\FF) = \WN(\RR_\alpha,\FF)$ but $
\WN_\bullet(\RR_\alpha,\GG,\FF) \not\subseteq \WN_\bullet(\RR_\alpha,\FF)$.\\
Use the command \autocom{WNo(S,G,F) = WNo(S,F)} for that purpose.

\subsubsection{Arbitraryness}
The approximated TRS is \emph{arbitrary} if there exists a term $t \in \TT(\FF)$
such that $t \to_{\RR_\alpha,\FF \cup \{ @ \}} @$.
If the approximated TRS $\RR_\alpha$ is arbitrary a witness term $t$ is given.
One can check whether the current approximated TRS $\RR_\alpha$ 
is arbitrary by using the \autocom{Arbitrary} command of the \autocom{Approx-TRS}
menu.

\subsection{Term Menu}

\subsubsection{Redex positions}
With the \autocom{Redexes} command, one may get the redexes positions
(according to the current TRS) of the current term.

\subsubsection{Needed redex positions}
With the \autocom{Needed redexes} command, one may get the needed-redexes 
positions
(according to the current approximation of the current TRS) of the current 
term.

\subsubsection{Normalizability}
With the \autocom{Normalizability} command, one may check whether a term
is normalizable with regard to the approximated TRS.\\
To check whether a redex in a term is needed in a term (according
to the current TRS and approximation), just enter the term where the
redex has been replaced with the {\tt o} symbol and check whether
it is normalizable.

\subsubsection{Recognizability}
With the \autocom{Recognizability} command of the \autocom{Term} menu, 
one may check whether the current
term is recognized by the current automaton.

\subsection{Termset}
A \emph{termset} is a set of (not necessarily ground) terms built upon
the current signature and set of variables.

\subsubsection{Accessibility}
With the \autocom{Accessibility} command, one may check whether a termset
is accessible from the current term with regard to the current
approximation.\\
To check accessibility to a single term, just enter a termset containing
that term.

\subsubsection{Termset automaton}
With the \autocom{Termset automaton} command, 
one may get obtain an automaton recognizing the termset.
With the \autocom{Accessibility} command one may check whether
the set of instances of the current termset is accessible from
the current term using the current approximation.

\subsection{Reduction and Needed Reduction}
The \textbf{Reduction} menu contains choices for reducing
the current term.
\subsubsection{Classical reduction}
One can perform a reduction step using either
the leftmost-outermost strategy or parallel-outermost strategy.
The ``Reduction to normal form'' command applies 
the parallel-outermost strategy
until a normal form is reached 
Use the \autocom{Stop} button (or command) to stop a looping computation.

\subsubsection{Needed reduction}
One can perform a needed-reduction step
using the leftmost-outermost needed redex.
The ``Needed reduction to normal form'' command applies the needed strategy
until a normal form is reached 
(may loop if term does not have a normal form).\\
Use the Stop button (or command) to stop a looping computation.

\subsection{TRS Automata}
The \autocom{TRS Automata} menu, gathers commands to compute
automata related to
the left-handsides of the current TRS or to its approximation.
The \autocom{Redex automaton}, \autocom{Reducible automaton},
\autocom{Nf automaton} compute the automata recognizing
the sets of redexes, reducible terms, normal forms.

We have the possibilities of computing Toyama's or Jacquemard's 
$\CC_{\RRa,\LL}$ automaton recognizing $(\to_\RRa^*)[\LL]$
with either \\
\begin{itemize}
\item $\LL = \NF$: command \autocom{Automaton C NF}, 
\item $\LL = L(\A_A)$: command \autocom{Automaton C A}, 
\item  $\LL = L(\A_{S})$: command \autocom{Automaton C S}, 
$S$ being the instances of the current termset.
\end{itemize}

We can also compute the automaton $\DD(\RRa,\FF)$ which
recognizes the set of reducible terms without $\RRa$-needed
redex: command \autocom{Automaton D}.

\subsection{Automata}
From this menu one can build new automata from existing ones. 
For instance
compute the complement of the current automaton 
(\autocom{Complement automaton}), complete the current automaton 
(\autocom{Complete automaton}), compute the intersection (or the union)
of the current automaton with an existing one (\autocom{Intersect automaton},
\autocom{Union automaton}).

\subsection{Automaton}
The operations of this menu apply to the current automaton.
Most of these operations check a property of the current automaton like:
\begin{itemize}
\item "is it deterministic?" (\autocom{Is Deterministic}),
\item "is it complete?" (\autocom{Is Complete}),
\item "is it minimal?" (\autocom{Is Minimal}),
\item "is it simplified (minimal uncompleted)?" (\autocom{Is Simplified}),
\item "does it recognize the empty language?" (\autocom{Empty automaton?}),
\item "is it included into another automaton" (\autocom{Inclusion automaton}), 
\item "is it equal (recognizes the same language) to another automaton" 
(\autocom{Equality automaton}), 
\item "is its intersection with another automaton empty"
(\autocom{Intersection emptiness}).

\end{itemize}

\section{How to exit if Autowrite crashes}
In that case, you should end up in the Lisp debugger. 
Just type \autocom{(cl::quit)} to exit Lisp.

%\section{Known bugs}

\section{Example of a session}
\noindent
\autocom{Command:} \autocombf{Load Spec}\\
\autocom{spec filename:} \autocombf{WRS.txt}\\
\autocom{Command:} \autocombf{Nf Automaton}\\
\autocom{Command:} \autocombf{Redex Automaton}\\
\autocom{Command:} \autocombf{Reducible Automaton}\\
\autocom{Command:} \autocombf{Complement Automaton}\\
\autocom{Command:} \autocombf{Equality Automaton}\\
\autocom{equality with automaton:} \autocombf{nf}\\
\autocom{Command:} \autocom{Retrieve automaton}\\
\autocom{automaton name:} \autocombf{redex}\\
\autocom{Command:} \autocom{Inclusion automaton}\\
\autocom{inclusion in automaton:} \autocombf{reducible}\\
\autocom{Command:} \autocom{Empty intersection?}\\
\autocom{emptyness of intersection with automaton:} \autocombf{nf}\\
\autocom{Command:} \autocom{Intersection automaton}\\
\autocom{Intersection with automaton:} \autocombf{nf}\\
\autocom{Command:} \autocom{Empty automaton?}\\
\autocom{Command:} \autocombf{Load Spec}\\
\autocom{spec filename:} \autocombf{S04.txt}\\
\autocom{Command:} \autocom{Call by need}\\
\autocom{Command:} \autocom{Approximation Nv}\\
\autocom{Command:} \autocom{Call by need}\\
\autocom{Command:} \autocom{Call by need Extra}\\
\autocom{Command:} \autocom{Redexes}\\
\autocom{Command:} \autocom{Needed redexes}\\
\autocom{Command:} \autocom{Automaton D}\\
\autocom{Command:} \autocom{Recognized by automaton}\\
\autocom{Command:} \autocom{Retrieve spec}\\
\autocom{spec name:} \autocombf{WRS.txt}\\
\autocom{Command:} \autocom{Termset automaton}\\
\autocom{Command:} \autocom{Accessibility automaton}\\
\autocom{Command:} \autocom{Needed reduction step}\\
\autocom{Command:} \autocom{Needed reduction to nf}\\
\autocom{Command:} \autocom{Retrieve term}\\
\autocom{term:} \autocombf{ *(*(0,s(0)),+(0,s(0)))}\\
\autocom{Command:} \autocom{Parallel outermost step}\\

\section{Snapshots of a session}
Figures~\ref{figure-sc1}, \ref{figure-sc2} and \ref{figure-sc3} at the
end of the document show some snapshots of the graphical interface.
\begin{figure}[ht]
\psfig{figure=sc1.ps}
\caption{Operations on the current term and the current automaton}
\label{figure-sc1}
\end{figure}
\begin{figure}[ht]
\psfig{figure=sc2.ps}
\caption{Boolean operations on automata}
\label{figure-sc2}
\end{figure}
\begin{figure}[ht]
\psfig{figure=sc3.ps}
\caption{Call by need queries}
\label{figure-sc3}
\end{figure}

\bibliography{references}
\end{document}
